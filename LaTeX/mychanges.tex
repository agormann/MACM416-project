\documentclass{myproject}

\graphicspath{{../Figures/}}

% title setup
\title{\vspace*{-1cm}Solution to the Inviscid Burgers' Equation by the Lax-Friedrichs Scheme\footnote{Placeholder title!}}
\date{}
\author{
    Andre Gormann\\
    agormann@sfu.ca
    \and
    Ethan MacDonald\\
    jem21@sfu.ca
}

% bibliography
\addbibresource{references.bib}

\renewcommand*{\thefootnote}{[\arabic{footnote}]}

\begin{document}

% title creation
\maketitle
\vspace*{-1cm}

% document 

\section{Introduction}

\section{Body}

The problem we will be considering is
\begin{equation}
    \begin{cases}
        & u_t + uu_x = 0 \qquad x \in [-L, L] \qquad t \geq 0 \\
        & u(x,0) = u_0(x) \\
        & u(-L,t) = u(L,t)
    \end{cases}
\end{equation}

We begin by discretizing the interval $[-L,L]$ by defining a mesh width $\Delta x = 2L/N$ so that 
\[
    x_j = -L + j\Delta x \qquad x_{j+1/2} = x_j + \frac{\Delta x}{2}
\]
We will also denote the time step by $\Delta t_n$ so that
\[
    t_n = n\Delta t_n
\]
The exact formula for $\Delta t_n$ will remain undefined for now as it must be a variable time-step.

We denote the \emph{pointwise values} of the true solution at the mesh point $(x_j, t_n)$ by 
\[
    u_j^n = u(x_j,t_n)
\]
and the \emph{cell average} of the true solution 
\[
    \bar{u}_j^n = \frac{1}{\Delta x} \int_{x_{j-1/2}}^{x_{j+1/2}} u(x,t_n) dx
\]

\[
    \int_{x_{j-1/2}}^{x_{j+1/2}} u(x,t_{n+1}) dx = \int_{x_{j-1/2}}^{x_{j+1/2}} u(x,t_{n}) dx - \left[ \int_{t_n}^{t_{n+1}} f(u(x_{j+1/2},t)) dx - \int_{t_n}^{t_{n+1}} f(u(x_{j-1/2},t)) dx \right]
\]

We say that a method is in \emph{conservation form} if 
\[
    U_j^{n+1} = U_j^n - \frac{\Delta t}{\Delta x} \left[ \mathcal{F}(U_{j}^{n}, U_{j+1}^{n}) - \mathcal{F}(U_{j-1}^{n}, U_{j+1}^{n}) \right]
\]

\[
    \bar{u}_j^{n+1} = \bar{u}_j^n - \frac{1}{\Delta x}\left[ \int_{t_n}^{t_{n+1}} f(u(x_{j+1/2},t)) dx - \int_{t_n}^{t_{n+1}} f(u(x_{j-1/2},t)) dx \right]
\]

So the numerical flux function $\mathcal{F}$ plays the role of an average flux through $x_{j\pm1/2}$ over the time interval $[t_n, t_{n+1}]$
\[
    \mathcal{F}(U_j^n, U_{j+1}^n) \sim \frac{1}{\Delta x} \int_{t_n}^{t_{n+1}} f(u(x_{j+1/2}, t)) dt \qquad \mathcal{F}(U_{j-1}^n, U_{j}^n) \sim \frac{1}{\Delta x} \int_{t_n}^{t_{n+1}} f(u(x_{j-1/2}, t)) dt
\]

\subsection{Lax-Friedrichs}

We have
\[
    U_j^{n+1} = \frac{1}{2}\left( U_{j-1}^{n} + U_{j+1}^{n} \right) - \frac{\Delta t}{2\Delta x}\left( f(U_{j+1}^{n}) - f(U_{j-1}^{n}) \right)
\]

We can write this in conservation form by taking
\[
    \mathcal{F}(U_j^n, U_{j+1}^n) := \frac{\Delta t}{2\Delta x}(U_j^n - U_{j+1}^n) + \frac{1}{2}\left( f(U_j^n) + f(U_{j+1}^n) \right)
\]


\section{Conclusion}

% bibliography
% \nocite{choksi2022}
\nocite{iserles2009}
% \nocite{kutz2013}
\nocite{trefethen2000}
% \nocite{learncfd}
% \nocite{evans2010}
\nocite{leveque1992}
\nocite{leveque2002}
\printbibliography

\end{document}